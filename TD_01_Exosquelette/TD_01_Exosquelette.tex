\documentclass[10pt,fleqn]{article} % Default font size and left-justified equations
\usepackage[%
    pdftitle={Modélisation systèmes multiphysiques : Modélisation par fonction de transfert et schéma-blocs},
    pdfauthor={Xavier Pessoles}]{hyperref}


\input{style/new_style}
\input{style/macros_SII}
\usepackage{multicol}
\usepackage{siunitx}

\fichetrue
%\fichefalse

\proftrue
%\proffalse

\tdtrue
%\tdfalse

\courstrue
\coursfalse


%\def\classe{\textsf{PSI$\star$ -- MP}}
\def\classe{\textsf{CMPT -- SI}}

\def\xxnumpartie{Cycle xx}
\def\xxpartie{Résoudre la loi Entrée -- Sortie du transmetteur d'un système}

\def\xxnumchapitre{Chap *** \vspace{.2cm}}
\def\xxchapitre{\hspace{.12cm} Résolution d'une loi ES }


\def\discipline{Sciences \\Industrielles de \\ l'Ingénieur}
\def\xxtete{Sciences Industrielles de l'Ingénieur}




\def\xxactivite{Activation 01}
\def\xxauteur{\textsl{Xavier Pessoles}}


\def\xxtitreexo{Assistance pour le maniement de charges dans l’industrie}
\def\xxsourceexo{\hspace{.2cm} \footnotesize{Concours Centrale Supelec TSI 2017}}




  
\def\xxposongletx{2}
\def\xxposonglettext{1.45}
\def\xxposonglety{20}
%\def\xxonglet{Part. 1 -- Ch. 3}
\def\xxonglet{\textsf{Cycle 01}}

\def\xxactivite{TD 01}
\def\xxauteur{\textsl{Xavier Pessoles}}

\def\xxcompetences{%
\vspace{-.5cm}
\footnotesize{
\textsl{%
\textbf{Savoirs et compétences :}\\
\vspace{-.2cm}
\begin{itemize}[label=\ding{112},font=\color{ocre}] 
\item\textit{Res2.C12} : loi entrée – sortie géométrique;
\item\textit{Res2.C12.SF1} : déterminer la loi entrée - sortie géométrique d’une chaîne cinématique;
\item\textit{Res2.C15} : loi entrée – sortie cinématique;
\item\textit{Res2.C15.SF1} : déterminer les relations de fermeture de la chaîne cinématique.
\end{itemize}}}}


\def\xxfigures{
\includegraphics[width=.3\textwidth]{images/fig_01}
}%figues de la page de garde


\def\xxpied{%
Cycle xx -- xx \\%dans le but de déterminer les contraintes géométriques dans les mécanismes\\% afin de valider leurs performances.\\
xx \xxactivite%
}

\setcounter{secnumdepth}{5}
%---------------------------------------------------------------------------


\begin{document}
%\chapterimage{png/Fond_Cin}
\input{style/new_pagegarde}
\vspace{4.5cm}
\pagestyle{fancy}
\thispagestyle{plain}


\def\columnseprulecolor{\color{ocre}}
\setlength{\columnseprule}{0.4pt} 

\ifprof
\else
\begin{multicols}{2}
\fi

\section*{Mise en situation -- Assurer le mouvement vertical}
\ifprof
\else

\noindent
\begin{tabular}{m{.6\linewidth}m{.3\linewidth}}
L’exosquelette est un appareil qui apporte à un être humain des capacités qu’il ne possède pas ou qu’il a perdues à cause d’un accident. Ce type d’appareil peut permettre à une personne de soulever des charges lourdes et diminuer considérablement les efforts à fournir sans la moindre fatigue. Après avoir revêtu un exosquelette adapté à sa morphologie et à sa taille, l’utilisateur peut faire ses mouvements en bénéficiant
d’une grande fluidité.
& 
\includegraphics[width=\linewidth]{images/fig_02}

\end{tabular}



\begin{center}
%\textit{}
\end{center}

\begin{center}
\includegraphics[width=\linewidth]{images/Exigences}
%\textit{}
\end{center}
\fi

%\subsection*{}

\begin{obj}
Proposer un modèle de connaissance des éléments réalisant l’exigence fonctionnelle <<~assurer le mouvement vertical~>> puis valider les performances attendues listées par le cahier des charges.
\end{obj}



\subsection*{Élaboration du modèle géométrique direct et du modèle articulaire inverse}
\begin{obj}
Élaborer la commande du moteur pilotant le genou à partir d’un mouvement défini dans l’espace
opérationnel puis converti dans l’espace articulaire.
\end{obj}


L’étude se limite au passage de la position accroupie à la position relevée de l’exosquelette. Lors de ce passage,
le point $O_2$ est en mouvement de translation verticale suivant la direction $\axe{O_0}{{z_0}}$ et sa vitesse de déplacement
évolue selon une loi trapézoïdale. Un modèle plan de la chaîne cinématique ouverte représente la partie inférieure
de l’exosquelette en position debout et fléchie.


\begin{center}
\includegraphics[width=\linewidth]{images/fig_03}
%\textit{}
\end{center}


%Selon le cahier des charges, pour assurer une bonne synchronisation des axes, l’exigence de précision statique suite à une entrée de type échelon, de type rampe ou de type accélération doit être inférieure à 1\%.


On donne le paramétrage du modèle proposé. 

\begin{center}
\includegraphics[width=\linewidth]{images/fig_04}
%\textit{}
\end{center}

\textbf{Hypothèses : }
\begin{itemize}
\item le référentiel lié au repère $\mathcal{R}_0\repere{A}{x_0}{y_0}{z_0}$ est galiléen et est fixe par rapport à la terre;
\item le point $O_2$ représentant la hanche se déplace verticalement selon la direction $\axe{O_0}{z_0}$;
\item l’angle $\alpha$ entre la charge transportée et la verticale $\vect{z_0}$ reste constant;
\item le point d’appui $A$ du pied sur le sol est considéré fixe par rapport à la terre;
\item lors du mouvement étudié la jambe (1) reste perpendiculaire au pied (3).
\end{itemize}

\textbf{Données : }
\begin{itemize}
\item $\theta_{10}=\angl{y_0}{y_1}=\angl{z_0}{z_1}$;
\item $\theta_{21}=\angl{y_1}{y_2}=\angl{z_1}{z_2}$;
\item $\alpha=\text{constante}$;
\item $L=\sqrt{\left(l_2+l_3\right)^2+l_4^2}$.
\end{itemize}

\subparagraph{}\textit{Déterminer littéralement les coordonnées opérationnelles $l_4$ et $h(t)$ en fonction des coordonnées articulaires $\theta_{10}$, $\theta_{21}$ et des paramètres dimensionnels $L$ et $l_1$.}
\ifprof
\begin{corrige}
On a $\vect{A O_1}+\vect{O_1 O_2}+\vect{O_2 O_0}+\vect{O_0 A}=\vect{0}$ soit 
$L\vect{y_1}+l_1\vect{y_2}-h(t)\vect{z_0}+l_4\vect{y_0}=\vect{0}$. 
En projetant sur $\vect{y_0}$ et $\vect{z_0}$ on a :
$$
\left\{
\begin{array}{l}
L\cos\theta_{10} +l_1\cos\left(\theta_{10}+\theta_{21}\right)+l_4={0} \\
L\sin\theta_{10}  +l_1\sin\left(\theta_{10}+\theta_{21}\right)-h(t) ={0} \\
\end{array}
\right.
$$

En projetant sur $\vect{y_1}$ et $\vect{z_1}$ on a :
$$
\left\{
\begin{array}{l}
L+l_1\cos\theta_{21}-h(t)\sin\theta_{10}+l_4\cos\theta_{10}={0} \\
l_1\sin\theta_{21}-h(t)\cos\theta_{10}-l_4\sin\theta_{10}={0}
\end{array}
\right.
$$


\end{corrige}
\else
\fi


\subparagraph{}\textit{Déterminer le modèle articulaire inverse $\theta_{10}$ et $\theta_{21}$ en fonction de $l_1$, $l_4$, $L$ et $h(t)$.}
\ifprof
\begin{corrige}
Pour exprimer $\theta_{10}$, on peut utiliser le premier système d'équation : 
$$
\left\{
\begin{array}{l}
L\cos\theta_{10}+l_4 =-l_1\cos\left(\theta_{10}+\theta_{21}\right) \\
L\sin\theta_{10}-h(t)  =-l_1\sin\left(\theta_{10}+\theta_{21}\right) \\
\end{array}
\right.
$$
En élevant les expressions au carré, on a alors : 
$
l_1^2 = \left(L\cos\theta_{10}+l_4 \right)^2+ \left(L\sin\theta_{10}-h(t)\right)^2
$
$
\Leftrightarrow 
l_1^2 = L^2+l_4^2 +h(t)^2+2Ll_4\cos\theta_{10} -2Lh(t)\sin\theta_{10}
$

$
\Leftrightarrow 
\dfrac{l_1^2 - L^2-l_4^2 -h(t)^2}{2L}=l_4\cos\theta_{10} -h(t)\sin\theta_{10}
$

En utilisant l'indication, on a : 
$$
\dfrac{l_4}{\sqrt{l_4^2 + h(t)^2}}\cos\theta_{10} +\dfrac{-h(t)}{\sqrt{l_4^2 + h(t)^2}}\sin\theta_{10}=\dfrac{l_1^2 - L^2-l_4^2 -h(t)^2}{2L{\sqrt{l_4^2 + h(t)^2}}}
$$
En conséquence, on pose $\cos\varphi=\dfrac{l_4}{\sqrt{l_4^2 + h(t)^2}}$ et 
$\sin\varphi = \dfrac{-h(t)}{\sqrt{l_4^2 + h(t)^2}}$. 
En conséquences %$\tan\varphi = \dfrac{\dfrac{-h(t)}{\sqrt{l_4^2 + h(t)^2}}}{\dfrac{l_4}{\sqrt{l_4^2 + h(t)^2}}}=\dfrac{-h(t)}{l_4}$.
$\tan\varphi =\dfrac{-h(t)}{l_4}$.


Par suite, $\cos\left( \theta_{10} - \varphi \right) =\dfrac{l_1^2 - L^2-l_4^2 -h(t)^2}{2L{\sqrt{l_4^2 + h(t)^2}}}$. On a donc 
$\theta_{10} =\arccos \left( \dfrac{l_1^2 - L^2-l_4^2 -h(t)^2}{2L{\sqrt{l_4^2 + h(t)^2}}}\right) + \varphi$.


Au final, 
$$\theta_{10} =\arccos \left( \dfrac{l_1^2 - L^2-l_4^2 -h(t)^2}{2L{\sqrt{l_4^2 + h(t)^2}}}\right) + \arctan\left(\dfrac{-h(t)}{l_4} \right).$$

Pour exprimer  $\theta_{21}$ on réutilise le premier système d'équations :
$
\left\{
\begin{array}{l}
L\cos\theta_{10} = - l_1\cos\left(\theta_{10}+\theta_{21}\right)-l_4 \\
L\sin\theta_{10} = - l_1\sin\left(\theta_{10}+\theta_{21}\right)+h(t) \\
\end{array}
\right.
$

On a alors 
$L^2 = l_1^2+l_4^2 +2 l_1l_4\cos\left(\theta_{10}+\theta_{21}\right)+h(t)^2-2h(t)l_1\sin\left(\theta_{10}+\theta_{21}\right)$.


\end{corrige}


\else
\fi


\begin{methode}
Lorsqu'on a une équation de la forme $A\cos\theta_{10}+B\sin\theta_{10}=C$. On peut normer cette équation en la mettant sous la forme $\dfrac{A}{\sqrt{A^2+B^2}}\cos\theta_{10}+\dfrac{B}{\sqrt{A^2+B^2}}\sin\theta_{10}=\dfrac{C}{\sqrt{A^2+B^2}}$.
On pose alors $\cos\varphi = \dfrac{A}{\sqrt{A^2+B^2}}$ et $\sin\varphi=\dfrac{B}{\sqrt{A^2+B^2}}$. On a alors $\cos\left( \theta_{10} - \varphi \right)=\dfrac{C}{\sqrt{A^2+B^2}}$.
\end{methode}


\subsection*{Élaboration du modèle cinématique}
\begin{obj}
En vue de dimensionner le moteur du genou, déterminer la vitesse articulaire en fonction de la  vitesse opérationnelle.
\end{obj}

\subparagraph{}\textit{Déterminer à partir du modèle articulaire inverse la vitesse angulaire 
$\theta_{21}$ en fonction de $h(t)$, $\dot{h}(t)$, $l_1$, $L$ et $\sin\theta_{21}$.}
\ifprof
\begin{corrige}
\end{corrige}
\else
\fi

Un modèle multiphysique a permis de déterminer les conditions suivantes correspondant à la vitesse maximale : $t=\SI{1,5}{s}$, $h(t=1,5)=\SI{0,829}{m}$, $\dot{h}(t=1,5)=\SI{0,422}{m.s^{-1}}$ et $\theta_{21} = 55,9\degres$. Les longueurs $l_1$ et $L$ valent
respectivement \SI{43,1}{cm} et \SI{51,8}{cm}. Le réducteur de vitesse utilisé a un rapport de réduction égal à $r=\dfrac{1}{120}$.

\subparagraph{}\textit{Déterminer la valeur maximale de la vitesse angulaire $\dot{\theta}_{21}$ et $\text{rad s}^{-1}$ puis celle de la fréquence de rotation d’un moteur de genou en $\text{tr min}^{-1}$.}
\ifprof
\begin{corrige}
\end{corrige}
\else
\fi



\footnotesize
\begin{tabular}{|p{.5\linewidth}|}
\hline
Éléments de corrigé :
\begin{enumerate}
\item ...
\end{enumerate} \\
\hline
\end{tabular}
\normalsize
\ifprof
\else
\end{multicols}
\fi

%\begin{center}
%\includegraphics[width=.65\linewidth]{images/cor_01}
%%\textit{}
%\end{center}

\end{document}

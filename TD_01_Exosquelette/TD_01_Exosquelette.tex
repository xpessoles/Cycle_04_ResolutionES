\documentclass[10pt,fleqn]{article} % Default font size and left-justified equations
\usepackage[%
    pdftitle={Modélisation systèmes multiphysiques : Modélisation par fonction de transfert et schéma-blocs},
    pdfauthor={Xavier Pessoles}]{hyperref}


\input{style/new_style}
\input{style/macros_SII}
\usepackage{multicol}
\usepackage{siunitx}

\fichetrue
%\fichefalse

\proftrue
\proffalse

\tdtrue
%\tdfalse

\courstrue
\coursfalse


%\def\classe{\textsf{PSI$\star$ -- MP}}
\def\classe{\textsf{CMPT -- SI}}

\def\xxnumpartie{Cycle xx}
\def\xxpartie{Résoudre la loi Entrée -- Sortie du transmetteur d'un système}

\def\xxnumchapitre{Chap *** \vspace{.2cm}}
\def\xxchapitre{\hspace{.12cm} Résolution d'une loi ES }


\def\discipline{Sciences \\Industrielles de \\ l'Ingénieur}
\def\xxtete{Sciences Industrielles de l'Ingénieur}




\def\xxactivite{Activation 01}
\def\xxauteur{\textsl{Xavier Pessoles}}


\def\xxtitreexo{Assistance pour le maniement de charges dans l’industrie}
\def\xxsourceexo{\hspace{.2cm} \footnotesize{Concours Centrale Supelec TSI 2017}}




  
\def\xxposongletx{2}
\def\xxposonglettext{1.45}
\def\xxposonglety{20}
%\def\xxonglet{Part. 1 -- Ch. 3}
\def\xxonglet{\textsf{Cycle 01}}

\def\xxactivite{TD 01}
\def\xxauteur{\textsl{Xavier Pessoles}}

\def\xxcompetences{%
\vspace{-.5cm}
\footnotesize{
\textsl{%
\textbf{Savoirs et compétences :}\\
\vspace{-.2cm}
\begin{itemize}[label=\ding{112},font=\color{ocre}] 
\item\textit{Res2.C12} : loi entrée – sortie géométrique;
\item\textit{Res2.C12.SF1} : déterminer la loi entrée - sortie géométrique d’une chaîne cinématique;
\item\textit{Res2.C15} : loi entrée – sortie cinématique;
\item\textit{Res2.C15.SF1} : déterminer les relations de fermeture de la chaîne cinématique.
\end{itemize}}}}


\def\xxfigures{
\includegraphics[width=.3\textwidth]{images/fig_01}
}%figues de la page de garde


\def\xxpied{%
Cycle xx -- xx \\%dans le but de déterminer les contraintes géométriques dans les mécanismes\\% afin de valider leurs performances.\\
xx \xxactivite%
}

\setcounter{secnumdepth}{5}
%---------------------------------------------------------------------------


\begin{document}
%\chapterimage{png/Fond_Cin}
\input{style/new_pagegarde}
\vspace{4.5cm}
\pagestyle{fancy}
\thispagestyle{plain}


\def\columnseprulecolor{\color{ocre}}
\setlength{\columnseprule}{0.4pt} 

%\ifprof

%\else
\begin{multicols}{2}
%\fi
\section*{Mise en situation -- Assurer le mouvement vertical}
\ifprof
\else

\noindent
\begin{tabular}{m{.6\linewidth}m{.3\linewidth}}
L’exosquelette est un appareil qui apporte à un être humain des capacités qu’il ne possède pas ou qu’il a perdues à cause d’un accident. Ce type d’appareil peut permettre à une personne de soulever des charges lourdes et diminuer considérablement les efforts à fournir sans la moindre fatigue. Après avoir revêtu un exosquelette adapté à sa morphologie et à sa taille, l’utilisateur peut faire ses mouvements en bénéficiant
d’une grande fluidité.
& 
\includegraphics[width=\linewidth]{images/fig_02}

\end{tabular}



\begin{center}
%\textit{}
\end{center}

\begin{center}
\includegraphics[width=\linewidth]{images/Exigences}
%\textit{}
\end{center}
\fi

%\subsection*{}

\begin{obj}
Proposer un modèle de connaissance des éléments réalisant l’exigence fonctionnelle <<~assurer le mouvement vertical~>> puis valider les performances attendues listées par le cahier des charges.
\end{obj}



\subsection*{Élaboration du modèle géométrique direct et du modèle articulaire inverse}
\begin{obj}
Élaborer la commande du moteur pilotant le genou à partir d’un mouvement défini dans l’espace
opérationnel puis converti dans l’espace articulaire.
\end{obj}


\ifprof
\else
L’étude se limite au passage de la position accroupie à la position relevée de l’exosquelette. Lors de ce passage,
le point $O_2$ est en mouvement de translation verticale suivant la direction $\axe{O_0}{\vect{z_0}}$ et sa vitesse de déplacement
évolue selon une loi trapézoïdale. Un modèle plan de la chaîne cinématique ouverte représente la partie inférieure
de l’exosquelette en position debout et fléchie.


\begin{center}
\includegraphics[width=\linewidth]{images/fig_03}
%\textit{}
\end{center}


%Selon le cahier des charges, pour assurer une bonne synchronisation des axes, l’exigence de précision statique suite à une entrée de type échelon, de type rampe ou de type accélération doit être inférieure à 1\%.
Le premier modèle défini figure suivante est adopté pour chaque axe.


\begin{center}
\includegraphics[width=\linewidth]{images/fig_05}
%\textit{}
\end{center}

\textbf{Notations : }
\begin{itemize}
\item  $\theta_{mC}(p)$ consigne de position de l’axe moteur (variable temporelle : $\theta_{mC}(t)$ en rad);
\item  $\theta_{m}(p)$ position de l’axe moteur (variable temporelle : $\theta_{m}(t)$ en rad);
\item  $C_{mC}(p)$ consigne de couple moteur (variable temporelle : $c_{mC}(t)$ en Nm);
\item  $C_{m}(p)$ couple moteur (variable temporelle : $c_{m}(t)$ en Nm);
\item  $C_{r}(p)$couple résistant perturbateur (variable temporelle : $c_{r}(t)$ en Nm);
\item  $K_1$ gain proportionnel du correcteur de l’asservissement de position (en $\text{s}^{-1}$);
\item  $\Omega_{mC}(p)$ consigne de vitesse de l’axe moteur (variable temporelle : $\Omega_{mC}(t)$ en $\text{rad s}^{-1}$);
\item  $\Omega_{m}(p)$ vitesse de l’axe moteur (variable temporelle : $\Omega_{m}(t)$ en  $\text{rad s}^{-1}$);
\item  $C_{\Omega}(p)$ correcteur de l’asservissement de vitesse;
\item  $M_C(p)$ modélise la boucle d’asservissement en couple de la machine électrique, considérée parfaite au vu de sa dynamique par rapport aux autres boucles : $M_C(p)=1$; 
\item  $J$ moment d’inertie de l’ensemble en mouvement, rapporté au niveau de l’axe moteur;
\item  $f$ coefficient de frottements visqueux équivalent pour l’ensemble en mouvement.
\end{itemize}


Le correcteur est de la forme : $C_{\Omega}(p)=K_2 \left( \dfrac{Jp +f}{Jp}\right)$. 

En utilisant le schéma-blocs précédent, on peut constater que : 
\begin{itemize}
\item l'écart est défini par la variable $\varepsilon(t) = \theta_{mC}(t)-\theta_m(t)$;
\item l'erreur entre l’entrée et la sortie est définie par la variable $\mu(t)= \theta_{mC}(t)-\theta_m(t)$.
\end{itemize}

Étant donné que le modèle utilisé est à retour unitaire, l’écart $\varepsilon(t)$ est égal à l’erreur $\mu(t)$. 
%La précision statique du système est définie par les paramètres suivants :
%\begin{itemize}
%\item $\varepsilon_p = \lim\limits_{t\to \infty} \varepsilon(t)$ suite à une entrée de type échelon unitaire $\theta_{mC}(t)=u(t)$, $\theta_{mC}(p)=\dfrac{1}{p}$, appelée erreur de position;
%\item $\varepsilon_v = \lim\limits_{t\to \infty} \varepsilon(t)$ suite à une entrée de type rampe unitaire $\theta_{mC}(t)=tu(t)$, $\theta_{mC}(p)=\dfrac{1}{p^2}$, appelée erreur de traînage;
%\item $\varepsilon_a = \lim\limits_{t\to \infty} \varepsilon(t)$ suite à une entrée de type accélération $\theta_{mC}(t)=\dfrac{t^2}{2}u(t)$, $\theta_{mC}(p)=\dfrac{1}{p^3}$, appelée erreur en accélération.
%\end{itemize}

\begin{hypo}
Le couple résistant évolue lentement au regard de la dynamique de l’asservissement, ce qui permet de considérer
pour la suite de l’étude $C_r(p)=0$.
\end{hypo}

\fi

\subparagraph{}\textit{Déterminer la grandeur physique de la consigne et la grandeur physique asservie à partir du modèle multiphysique présenté plus bas et préciser leurs unités de base dans le système international d’unités (SI).}
\ifprof
\begin{corrige}~\\
Il s'agit d'un asservissement en position. 

\begin{center}
\includegraphics[width=\linewidth]{images/cor_01}
\end{center}
\end{corrige}

\else
\fi


\subparagraph{}\textit{Exprimer $H_{\Omega}(p)=\dfrac{\Omega_m(p)}{\Omega_{mC}(p)}$
en fonction de $J$, $K_2$ et $p$.}
\ifprof
\begin{corrige}
En faisant l'hypothèse que le couple perturbateur est nul, on a :
$H_{\Omega}(p)=\dfrac{\Omega_m(p)}{\Omega_{mC}(p)}=\dfrac{C_{\Omega}(p)M_C(p)\dfrac{1}{Jp+f}}{1+C_{\Omega}(p)M_C(p)\dfrac{1}{Jp+f}}$. En conséquences : 
$H_{\Omega}(p)=\dfrac{C_{\Omega} K_2}{Jp+C_{\Omega} K_2 } = \dfrac{1}{\dfrac{Jp}{C_{\Omega} K_2}+1 } $.

\end{corrige}
\else
\fi

\subparagraph{}\textit{Exprimer $\varepsilon(p)$ en fonction de $\theta_{mC}(p)$, $H_{\Omega}(p)$, $K_1$ et $p$.}
\ifprof

\begin{corrige}
D'une part, $\varepsilon(p)=\theta_{mC}(p)-\theta_{m}(p)$. D'autre part, 
$\theta_{m}(p) =H_{\Omega}(p) \dfrac{K_1}{p} \varepsilon(p)$. Par suite, 
$\varepsilon(p)=\theta_{mC}(p)-H_{\Omega}(p) \dfrac{K_1}{p}\varepsilon(p) $ 
$\Leftrightarrow \varepsilon(p)\left( 1+H_{\Omega}(p) \dfrac{K_1}{p}\right)= \theta_{mC}(p)$. 
En conséquences, $\varepsilon(p)=\dfrac{ \theta_{mC}(p)}{ 1+H_{\Omega}(p) \dfrac{K_1}{p}}$.
\end{corrige}
\else
\fi


\subparagraph{}\textit{Déterminer l’erreur de position $\varepsilon_p$ puis l’erreur de traînage $\varepsilon_v$. Conclure sur la valeur de $K_1$ pour satisfaire
à l’exigence d’erreur en traînage.}

\ifprof
\begin{corrige}
On a :
\begin{itemize}
\item $\varepsilon_p = \lim\limits_{t\to \infty} \varepsilon(t)= \lim\limits_{p\to 0} p\varepsilon(p) $ $= \lim\limits_{p\to 0} p \dfrac{ 1}{ 1+H_{\Omega}(p) \dfrac{K_1}{p}} \dfrac {1}{p}$
$= \lim\limits_{p\to 0} \dfrac{ 1}{ 1+\dfrac{1}{\dfrac{Jp}{C_{\Omega} K_2}+1 } \dfrac{K_1}{p}} = 0$ (ce qui était prévisible pour un système de classe 1);
\item $\varepsilon_v = \lim\limits_{t\to \infty} \varepsilon(t)= \lim\limits_{p\to 0} p\varepsilon(p) $ $= \lim\limits_{p\to 0} p \dfrac{ 1}{ 1+H_{\Omega}(p) \dfrac{K_1}{p}} \dfrac {1}{p^2}$
$= \lim\limits_{p\to 0} \dfrac{ 1}{ 1+\dfrac{1}{\dfrac{Jp}{C_{\Omega} K_2}+1 } \dfrac{K_1}{p}}\dfrac {1}{p} $
$= \lim\limits_{p\to 0} \dfrac{ 1}{ p+\dfrac{1}{\dfrac{Jp}{C_{\Omega} K_2}+1 } K_1}= \dfrac{1}{K_1}$ (ce qui était prévisible pour un système de classe 1 et de gain $K_1$ en BO).
\end{itemize}

Ainsi, pour avoir une erreur de traînage inférieure à 1\%, il faut $\dfrac{1}{K_1}<0,01$ et $K_1 >100$.
\end{corrige}
\else
\fi







\subparagraph{}\textit{Déterminer l’erreur en accélération et conclure quant au respect du cahier des charges.}


\ifprof

\begin{corrige}
En raisonnant de même, on a :
 $\varepsilon_a = \lim\limits_{t\to \infty} \varepsilon(t)= \lim\limits_{p\to 0} p\varepsilon(p) $ $= \lim\limits_{p\to 0} p \dfrac{ 1}{ 1+H_{\Omega}(p) \dfrac{K_1}{p}} \dfrac {1}{p^3}$
$= \lim\limits_{p\to 0} \dfrac{ 1}{ 1+\dfrac{1}{\dfrac{Jp}{C_{\Omega} K_2}+1 } \dfrac{K_1}{p}}\dfrac {1}{p^2} = 0$
$= \lim\limits_{p\to 0} \dfrac{ 1}{ p^2+\dfrac{p}{\dfrac{Jp}{C_{\Omega} K_2}+1 } K_1}= \infty$ (ce qui était prévisible pour un système de classe 1).


Ainsi, le correcteur choisi ne permet pas de vérifier le cahier des charges. 
\end{corrige}
\else
\fi

\ifprof
\else
Pour satisfaire l’exigence d’une erreur en accélération inférieure à 1\%, le second modèle avec anticipation de la
vitesse est adopté avec $H_{\Omega}(p)=\dfrac{1}{1+Tp}$ et $T=\SI{33}{ms}$.

\begin{center}
\includegraphics[width=\linewidth]{images/fig_08}
%\textit{}
\end{center}
\fi

\subparagraph{}\textit{Exprimer $\varepsilon(p)$ en fonction de $\theta_{mC}(p)$, $T$, $K_1$, $K_3$ et $p$.}
\ifprof

\begin{corrige}
En utilisant le schéma-blocs, on a : 
\begin{itemize}
\item 
$ \varepsilon(p)=\theta_{mC}(p)-\theta_{m}(p)$;
\item  $\Omega_{mC}(p)=K_3 p \theta_{mC}(p) + K_1 \varepsilon(p)$;
\item $\theta_m(p)=\Omega_{mC}(p) \dfrac{1}{p}\dfrac{1}{1+Tp}$. 
\end{itemize}
On a donc : 
$ \varepsilon(p)=\theta_{mC}(p)-\Omega_{mC}(p) \dfrac{1}{p}\dfrac{1}{1+Tp}$ 
$= \theta_{mC}(p)- 
\left(K_3 p \theta_{mC}(p) + K_1 \varepsilon(p)  \right)
\dfrac{1}{p \left( 1+Tp\right)}$
$= \theta_{mC}(p)- 
 \dfrac{K_3 p }{p \left( 1+Tp\right)} \theta_{mC}(p)
-  \dfrac{K_1 }{p \left( 1+Tp\right)} \varepsilon(p)
$. 

On a alors 
$
\varepsilon(p)  \left(1+ \dfrac{K_1 }{p \left( 1+Tp\right)} \right)
=  \theta_{mC}(p)\left(1-  \dfrac{K_3}{ 1+Tp}\right)$

$\Leftrightarrow
\varepsilon(p)  \dfrac{p \left( 1+Tp\right)+K_1 }{p \left( 1+Tp\right)} 
=  \theta_{mC}(p)  \dfrac{ 1+Tp-K_3  }{ 1+Tp}
$.

Enfin,  
$\varepsilon(p) = \theta_{mC}(p)\dfrac{p\left( 1+Tp-K_3\right) }{p \left( 1+Tp\right)+K_1}
$.
\end{corrige}

\else
\fi

Le second modèle avec anticipation de la figure précédente n’a pas d’incidence sur la valeur de l’erreur de position.

\subparagraph{}\textit{Exprimer l’erreur de traînage et déterminer la valeur de $K_3$ permettant l’annuler cette erreur.}

\ifprof

\begin{corrige}
$\varepsilon_v = \lim\limits_{t\to \infty} \varepsilon(t)= \lim\limits_{p\to 0} p\varepsilon(p) $ 
$= \lim\limits_{p\to 0} p \dfrac{p\left( 1+Tp-K_3\right) }{p \left( 1+Tp\right)+K_1} \dfrac {1}{p^2}$
$= \lim\limits_{p\to 0}  \dfrac{\left( 1+Tp-K_3\right) }{p \left( 1+Tp\right)+K_1}$
$=\dfrac{1-K_3}{K_1}$.


Au final, pour annuler l'erreur de traînage, on doit avoir $K_3=1$.
\end{corrige}
\else
\fi
\subparagraph{}\textit{Exprimer et déterminer l’erreur d’accélération en prenant les valeurs de $K_3$ et de $K_1$ déterminées
précédemment. Conclure quant au respect du cahier des charges.}

\ifprof

On a : 

\begin{corrige}
$\varepsilon_a = \lim\limits_{t\to \infty} \varepsilon(t)= \lim\limits_{p\to 0} p\varepsilon(p) $ 
$= \lim\limits_{p\to 0} p \dfrac{p\left( 1+Tp-K_3\right) }{p \left( 1+Tp\right)+K_1} \dfrac {1}{p^3}$
$= \lim\limits_{p\to 0} \dfrac{\left( 1+Tp-K_3\right) }{p \left( 1+Tp\right)+K_1} \dfrac {1}{p}$. En prenant $K_3=1$ et $K_1=100$, on obtient :
$\varepsilon_a = \dfrac{ T}{p \left( 1+Tp\right)+100}=\dfrac{33\times 10^{-3}}{100} $.
L'erreur est donc de $33\times 10^{-5}$. Le cahier des charges est donc validé. 

\end{corrige}
\else
\fi

\subsection*{Synthèse}
\subparagraph{}
\textit{En utilisant la figure ci-dessous, conclure sur les actions qui ont mené à une validation du cahier des charges.}

\footnotesize
\begin{tabular}{|p{\linewidth}|}
\hline
Eléments de corrigé :
\begin{enumerate}
\item Asservissement en position.
\item $H_{\Omega}(p)= {1}/\left(\dfrac{Jp}{C_{\Omega} K_2}+1 \right) $.
\item $\varepsilon(p)=\left( \theta_{mC}(p)\right)/\left( 1+H_{\Omega}(p) \dfrac{K_1}{p}\right)$
\item $\varepsilon(p)= 0, \varepsilon_v = \dfrac{1}{K_1}$   et $K_1 >100$.
\item $\varepsilon_a = \infty$.
\item $\varepsilon(p) = \theta_{mC}(p)\left(p\left( 1+Tp-K_3\right) \right)/\left(p \left( 1+Tp\right)+K_1\right) $.
\item $\varepsilon_v =\dfrac{1-K_3}{K_1}$, $K_3=1$.
\item $\varepsilon_a = \dfrac{33\times 10^{-3}}{100} $. Le cahier des charges est donc validé. 
\end{enumerate} \\
\hline
\end{tabular}
\normalsize
\end{multicols}
\ifprof
\begin{center}
\includegraphics[width=.8\linewidth]{images/cor_02}
%\textit{}
\end{center}

\else
\begin{center}
\includegraphics[width=\linewidth]{images/fig_10}
%\textit{}
\end{center}
\fi

%\begin{center}
%\includegraphics[width=.65\linewidth]{images/cor_01}
%%\textit{}
%\end{center}

\end{document}
